\newpage

\section{Coordenação}

Coordenação ou participação em grupos de pesquisa onde o candidato exerceu papel de liderança.


\subsection{Coordenação de Projeto de Pesquisa Internacional}

\begin{itemize}

\item \textbf{Coordenação do \textbf{\emph{TriggerEgamma}}:} Eleito pela colaboração, através de eleição fechada, para o cargo de coordenação (\emph{co-convener}) do grupo responsável pelo trigger de elétrons e fótons do experimento ATLAS, em conjunto com o pesquisador Christopher John Meyer (\emph{convener}), da Universidade de \emph{Indiana Bloomington}, USA, para coordenar os desenvolvimentos relacionados ao sistema de \emph{trigger} para a \emph{Run-3}. A posição teve início no dia \textbf{01 de abril de 2021} e terminou em \textbf{31 de maio de 2022}. Durante a coordenação do grupo, os seguintes trabalhos foram realizados: Participação em reuniões de coordenação com os outros grupos de trabalho do CERN; Participação em reuniões de trabalho semanais com objetivo de organizar os trabalhos e guiar os grupos de pesquisa para atender as demandas do CERN e orientações individuais com cada membro do grupo de pesquisa. Esta posição teve forte influência dos trabalhos desenvolvidos pela tese de doutorado do candidato.

\end{itemize}


\subsection{Coordenação de Projetos Nacionais}

\begin{itemize}

\item \textbf{Coordenação de Ensaios Técnicos e Certificação de Módulos Criptográficos:}
Coordenação do grupo técnico de ensaios criptográficos do Laboratório de Aplicações Tecnológicas para o Setor Produtivo de Industrial (LASPI) da UFRJ, sob a coordenação do Professor Carlos José Ribas D'Avilla, de \textbf{fevereiro de 2016} até \textbf{dezembro de 2021}. Durante a coordenação, o candidato foi responsável pelo gerenciamento e orientação da equipe técnica do laboratório, elaboração dos planos de ensaio técnico para módulos de segurança criptográficos, responsável técnico pela emissão dos relatórios e pareceres e encarregado do grupo de trabalho responsável pela homologação da biblioteca criptográfica e do \textit{hardware}/\textit{software} das Urnas Eletrônicas (Modelo 2020) utilizadas nas eleições para presidência da república do ano de 2022. Atualmente, o LASPI é o único laboratório autorizado, pelo INMETRO e pela presidência da república, a realizar ensaios técnicos em módulos de segurança no país.

\end{itemize}

%\subsection{Coordenação de Projeto de Pesquisa Nacional}
