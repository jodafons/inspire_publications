

\section{Conteúdo Midiático}

Conteúdos de mídia produzidos que tiveram influência dos trabalhos desenvolvidos pela tese de doutorado do candidato.

\begin{itemize}


\item \textbf{``Aluno da COPPE é escolhido para coordenar área do ATLAS'', publicado em 3 de março de 2021, 
    Planeta COPPE Notícias},
    O aluno de doutorado da COPPE/UFRJ, João Victor da Fonseca, foi escolhido para 
    coordenar o Egamma Trigger Signature, uma importante área do ATLAS, maior experimento 
    de detecção de partículas do Large Hadron Collider (LHC), o acelerador de partículas 
    do CERN.  Aluno do Programa de Engenharia Elétrica (PEE) da COPPE, João dividirá a 
    função com o pesquisador Chris Meyer (Indiana University Bloomington, EUA). 

    Segundo o professor da COPPE, José Manoel de Seixas, coordenador da equipe de 
    pesquisadores brasileiros que atuam no ATLAS, João Victor foi convidado para 
    assumir a área devido ao ``trabalho magnífico que tem feito e pelo histórico 
    de contribuição da COPPE na filtragem \emph{online} de elétrons. Ele assume o grupo 
    de assinatura no sistema de filtragem \emph{online}, uma área importante que busca 
    detectar se há elétrons e fótons nos decaimentos das colisões de partículas, e 
    cobre larga quantidade de canais físicos de interesse, do bóson de Higgs, de 
    supersimetria, e matéria escura.'', explica Seixas.

    Além de João Victor, um ex-aluno da COPPE, o pesquisador Denis Damazio, que atua no 
    CERN desde 2005, como pesquisador do Brookhaven National Laboratory (EUA), coordenará 
    a partir de julho o HLT Calo Software, juntamente com Jochen Jens Heinrich. O HLT 
    Calorimeter é o calorímetro do sistema de filtragem de alto nível.
    ``O calorímetro mede a energia das partículas incidentes nas colisões e fornece 
    uma resposta mais rápida do que o sistema de precisão, baseado em imagens. 
    A informação obtida pelo calorímetro, que tem 200 mil canais de leitura, alimenta o 
    NeuralRinger, sistema de filtragem \emph{online} desenvolvido na COPPE'', explica o 
    professor José Manoel de Seixas.

    O CERN é o mais importante laboratório de física de partículas do mundo. Localizado 
    entre França e Suíça, o reúne 12 mil pesquisadores de mais de 100 nacionalidades, 
    dos quais 131 são brasileiros, e cuja principal missão é descobrir a origem do 
    universo. O laboratório europeu é responsável pela criação do protocolo www, aceito 
    internacionalmente como padrão para navegação na internet, e pela descoberta do 
    bóson de Higgs, conhecida como "a partícula de Deus", a qual permite que matéria 
    tenha massa, e que rendeu o Prêmio Nobel de Física aos cientistas Peter Higgs e 
    François Englert em 2013.

    \item \textbf{``ATLAS adota sistema desenvolvido pela COPPE'', publicado em 7 de março de 2018,
    Planeta COPPE Notícias}, Um sistema de filtragem \emph{online} de elétrons desenvolvido por pesquisadores 
    da COPPE/UFRJ foi escolhido como referência para ser utilizado pelo 
    ATLAS, experimento de detecção de partículas instalado no CERN, o laboratório 
    que investiga a origem do universo. Denominado NeuralRinger, o sistema 
    possibilitará novas descobertas com menor custo financeiro para o CERN, que 
    no momento está ampliando o número de choques entre prótons para aumentar 
    os eventos físicos, essenciais à investigação e descoberta de possíveis 
    novas partículas.

    O sistema desenvolvido no Brasil permite decidir a cada 10 milissegundos 
    quais informações reter dentre os mais de 60 Terabytes de informação geradas 
    a cada segundo nas passagens de feixes de partículas conduzidas no laboratório. 
    Os pesquisadores do CERN querem aumentar o número de eventos por colisão de 25 
    para 200, até 2024, o que aumentaria exponencialmente o volume de dados de 
    interesse científico gerados. Criado dentro do conceito de redes neurais, 
    o NeuralRinger permite encontrar as ``agulhas'' (eventos físicos de interesse) 
    neste "palheiro" que não para de crescer.

    ``A expectativa para 2018 é que o número de eventos por colisões salte de 
    25 para 88, sendo fundamental a calibração do algoritmo para a reconstrução 
    dos eventos selecionados pela filtragem \emph{online}, uma etapa muito importante 
    da calorimetria'', ressalta o professor de Engenharia Elétrica da COPPE, 
    José Manoel de Seixas, coordenador da equipe brasileira no ATLAS.
    
    No último mês de dezembro, um projeto de pesquisa visando o aperfeiçoamento 
    do algoritmo do NeuralRinger foi aprovado pela Coordenação de Aperfeiçoamento 
    de Pessoal de Nível Superior (Capes) e pelo Comitê Francês de Avaliação da 
    Cooperação Universitária com o Brasil (COFECUB). O edital prevê o intercâmbio 
    de pesquisadores da COPPE, da Université Paris VI (Pierre e Marie Curie) e 
    da Université Clermont-Ferrand (Blaise Pascal), com duração de quatro anos 
    (de 2018 a 2021). 
    
    O ATLAS, que em outubro de 2017 completou 25 anos de existência, tem tido um 
    importante papel em descobertas recentes. Foi realizada no ATLAS a pesquisa 
    que  detectou, pela primeira vez em um experimento com alta energia, o fenômeno 
    da dispersão da luz pela luz, previsto pela teoria quântica, porém negado pela 
    teoria eletromagnética. O experimento também teve uma relevante contribuição 
    na descoberta do bóson de Higgs - a chamada "partícula de Deus". A comprovação 
    da existência desta partícula rendeu aos cientistas Peter Higgs e François 
    Englert o Prêmio Nobel de Física de 2013.

    Segundo o professor Seixas, a grande vantagem do NeuralRinger é que a filtragem, 
    realizada \emph{online}, já possibilita decidir se a informação é potencialmente útil, 
    reduzindo a demanda computacional para coletar e preservar este enorme volume de 
    informação. O sistema desenvolvido na COPPE também poupa muitos recursos de filtragem 
    \emph{online}, porque ele reduz entre 2 a 6 vezes a demanda por processamento de dados, 
    dependendo da região do detector e da energia. "Se você jogar fora uma informação 
    gerada neste processo, você nunca mais irá resgatá-la, por isso a filtragem é 
    um processo de escolha muito sensível e muito importante", alerta Seixas.
 
    De acordo com o professor Seixas, o processo de reconhecimento de partículas 
    se dá por reconhecimento de padrão. Ao multiplicar o número de colisões, o volume 
    de informações aumentaria exponencialmente, exigindo a ampliação da ``farm'', 
    jargão utilizado pelos pesquisadores referindo-se aos milhares de computadores 
    utilizados para processar milhões de informações geradas pelo ATLAS.
     
    ``Estamos desenvolvendo esse sistema, baseado em redes neurais, desde 1991. 
    Ele engloba vinte redes neurais que operam simultaneamente para cada seção 
    do detector de partículas, permitindo o uso mais eficiente do 
    calorímetro (absorve a energia das partículas), e menor utilização do tracking, 
    seção do ATLAS que exige mais capacidade computacional'', explica Seixas.

\end{itemize}
