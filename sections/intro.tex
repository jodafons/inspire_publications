


\section{Apresentação}\label{sec:apresentation}

O presente candidato é natural do Estado do Rio de Janeiro, nascido 
e criado em Irajá, Zona Norte da cidade do Rio de Janeiro, onde 
reside grande parte de sua família. Em 2009, concluiu o ensino médio no 
Colégio Pedro II, unidade São Cristóvão e, em junho de 2016, graduou-se em Engenharia 
Eletrônica e de Computação pela Universidade Federal do Rio de Janeiro (UFRJ).



Durante a graduação participou intensamente do programa de iniciação 
cientifica (IC), vinculado ao Laboratório de Processamento de Sinais (LPS) da 
COPPE (Instituto Alberto Luiz Coimbra de Pós-Graduação e Pesquisa em Engenharia) 
sob a coordenação do Professor José Manoel de Seixas. No início da IC (2012), participou 
do projeto responsável pela construção de um sistema de triagem, baseado em aprendizado de 
máquina, para pacientes com tuberculose no hospital Universitário Clementino Fraga Filho da UFRJ. 
O trabalho desenvolvido pelo candidato fez parte da tese de doutorado na Faculdade
de Medicina da UFRJ~\cite{tese_fabio_aguiar} do responsável pelo projeto na área da saúde, 
tendo sido publicado em 2016 em~\cite{paper_tb_som}. 

Em 2014, o candidato ingressou no projeto de filtragem \emph{online} de elétrons 
desenvolvido em parceria entre a COPPE, desde 1992, e o experimento A Toroidal LHC 
ApparatuS (ATLAS), que é o maior experimento acoplado ao Grande Colisor de Hádrons ou 
Large Hadron Collider (LHC), pertencente ao Centro Europeu de Pesquisas Nucleares (CERN) 
localizado na fronteira entre a França e a Suíça. Nesta época, o projeto havia sido parte 
integrante de diversos trabalhos de graduação, mestrado e doutorado capitaneados pela 
COPPE (a partir do Professor José Manoel de Seixas e o Professor Luiz Pereira Calôba), desde 
os anos 90, mas estas propostas alternativas não haviam sido escolhidas para operação no 
experimento, realizando a seleção de candidatos à elétrons durante as colisões. 



Originalmente, o projeto previa duas inserções de grande impacto na colaboração. 
A primeira se utilizava da estrutura cônica do desenvolvimento do chuveiro para montar 
as variáveis discriminantes (padrão não utilizado pela colaboração) e por fim, essas 
variáveis eram combinadas com um classificador baseado em redes neurais artificias para 
tomar a decisão de aceitar ou não o candidato à elétron. Até então, esses dois pontos eram 
objetos de intensa discussão, por parte da colaboração. Em especial este último, que 
introduzia uma técnica de seleção, baseada em Aprendizado de Máquina, pouco explorada pela 
colaboração em ambiente \emph{online} (mesmo em 2014). 



No final de 2014, ainda na graduação, o candidato foi enviado ao CERN para reestruturar o sistema 
responsável pela construção das variáveis e tomadas de decisão. Além disso, a viagem, com duração 
de 6 meses, teve como missão construir capital político-científico entre o grupo do 
Brasil (COPPE) e possíveis pesquisadores internacionais, necessário para o avanço da 
proposta. Com o objetivo de fortalecer a proposta junto ao subgrupo responsável pela filtragem 
\emph{online} de elétrons, o candidato optou por participar de diferentes frentes, além 
daquelas que viriam fazer parte do seu trabalho de final de curso e doutorado, dentro do 
grupo responsável pela manutenção, concepção e análise de todo o sistema de filtragem de 
elétrons e fótons do ATLAS (chamado de \emph{Trigger e/gamma}).



Dentre os trabalhos desenvolvidos neste período, entre 2014 e 2016, em nome da UFRJ/COPPE, 
destacam-se: o desenvolvimento do protótipo do novo sistema de filtragem \emph{online} de elétrons 
baseados em redes neurais para a etapa rápida apresentado na \emph{17th International workshop on 
Advanced Computing and Analysis Techniques in Physics reseach (ACAT 2016)}~\cite{acat2016} no Chile, o 
desenvolvimento do primeiro sistema de emulação de cadeias de elétrons e fótons em ambiente de 
análise da colaboração e o desenvolvimento do sistema de monitoramento do sistema de 
filtragem \emph{online} para elétrons e fótons. Ainda foram apresentados diversos trabalhos em 
congressos, como os Encontros Nacionais de Física de Partículas e Campos 
organizado pelo Brasil neste período.
Em 2016, o candidato ingressou no curso de mestrado em inteligência computacional do 
Programa de Engenharia Elétrica (PEE) da COPPE/UFRJ, enquanto projeto de pesquisa associado 
ao Laboratório de Processamento de Sinais, onde continuou o desenvolvimento do sistema de 
filtragem \emph{online} de elétrons baseados em redes neurais artificiais e formatação dos sinais 
do calorímetro (medição de energia) em anéis concêntricos de deposição de energia, agora 
numa formulação em \emph{ensemble}, com 20 modelos neurais operando em termos de posição de impacto 
da colisão e na energia transversa do subproduto resultante de uma colisão.


Após 2 anos de interação com a colaboração, a proposta desenvolvida pelo candidato passou 
a contar com o apoio de diversos pesquisadores do grupo internacional (Chile, Argentina, França e Estados Unidos). 
Ainda, no final de 
2016, com a previsão de colapso do sistema de filtragem do ATLAS devido ao crescimento da 
taxa de eventos processados, para o próximo ano de colisão no LHC, a proposta capitaneada 
pela COPPE ganhou força. No início de 2017, a proposta foi finamente aceita, após quase 
25 anos de trabalhos nessa área pelo Brasil, pelo ATLAS \emph{Trigger General Group} (grupo geral responsável 
pelo gerenciamento do sistema de filtragem do ATLAS) e entrou em comissionamento no início de 
junho de 2017. Com o sucesso da operação e a redução nas taxas de entrada do sistema de 
filtragem para elétrons, o comissionamento foi aprovado e a técnica passou a ser responsável 
pela filtragem de elétrons no segundo estágio do \emph{Trigger}. Deve-se destacar a importância do 
trabalho de pesquisa colaborativa envolvendo diferentes níveis de especialização da filtragem 
\emph{online} do ATLAS e na enorme experiência em pesquisa que essa atividade produziu no 
candidato, bem como a responsabilidade científica reforçada pela atuação brasileira no 
ATLAS, consolidando a forte participação do \emph{Cluster} ATLAS/Brazil no ambiente de seleção 
de eventos pela primeira vez após mais de duas décadas de atuação no experimento.



Após o comissionamento do projeto, considerando-se a amplitude da proposta de pesquisa em 
desenvolvimento, o candidato foi convidado pelo PEE/COPPE para seguir diretamente para o 
doutorado, sem a necessidade de defesa da dissertação de mestrado. Assim, foi encaminhado 
um processo junto a COPPE/PEE (Proc.053728/09-00) para ser aprovado para o Doutorado sem 
defesa de dissertação de mestrado. Para ser aprovado neste processo, o requerente deve 
possuir um coeficiente de rendimento superior à 2,5/3,0 ao completar todas as cadeiras do 
curso de mestrado e possuir uma contribuição de grande relevância para a ciência do Brasil. 
O processo foi aprovado e no terceiro trimestre de 2017, o presente candidato passou a realizar 
as atividades do curso de doutorado no PEE/COPPE. Em novembro de 2017, o candidato foi convidado 
para falar em nome de todo o grupo \emph{Trigger e/gamma} no \emph{e/gamma Workshop} 
(encontro internacional) em Hamburgo, Alemanha.



Ainda naquele ano, o candidato entrou com o pedido para tornar-se autor do experimento ATLAS. 
Para tornar-se autor do ATLAS, o requerente deve propor uma solução de grande impacto na 
colaboração e defendê-la, ao final de um ano (doze meses) para todos os grupos relacionados 
à proposta no ATLAS.  Caso seja aprovado, considera-se que houve uma contribuição relevante 
para toda a colaboração e consequente aquisição de dados de forma eficiente e o requerente 
passa a compor a lista de autores do experimento e automaticamente é incluído em todos os 
artigos e trabalhos produzidos, a partir daquela data, pelo ATLAS. 



Como proposta foi selecionada como tarefa de qualificação (\emph{Qualification Task}) o treinamento 
e avaliação do \emph{ensemble} de redes neurais artificias a partir de dados reais coletados na 
segunda metade de 2017 para o início das operações no último ano de colisão da \emph{Run 2} (2018). 
A proposta de projeto foi submetida pelo candidato em outubro de 2017 e em junho de 2018 
foi aprovada pelo comitê. Atualmente, o candidato conta com 215 artigos~\cite{my_inspirehep, my_orcid} publicados com toda 
a colaboração ATLAS nos mais diferentes veículos de comunicação científica. Em 2018, o candidato 
foi novamente convidado para apresentar, em nome de toda a colaboração ATLAS, os resultados 
referentes ao novo sistema de filtragem \emph{online} para elétrons baseado em redes neurais~\cite{calor2018}
na \emph{18th International Conference on Calorimetry in Particle Physics} em Eugene, Oregon, Estados 
Unidos, que é a principal conferência internacional em calorimetria de altas energias.



Em janeiro de 2020, o candidato fundou, em conjunto com o pesquisador e, naquela 
época aluno de pós-doutorado pela Universidade Pierre Maria Curie (Sorbonne, França), Werner 
Spolidouro Freund (doutor pelo PEE/COPPE), o grupo responsável pela contrução de um \emph{framework} 
de simulação de eventos em um 
calorímetro genérico baseado nos esquemas técnicos do experimento ATLAS.  O objetivo da iniciativa 
era construir um ambiente público de geração de eventos de simulação (similar àqueles utilizados 
pelo projeto de filtragem \emph{online} de elétrons) de fácil uso e adaptação por parte dos alunos e 
professores. Com o processo de \emph{upgrade} já agora vivido pelo Experimento ATLAS e com a preparação 
dos futuros experimentos de altas energias, que vêm recebendo crescente atenção da comunidade 
científica internacional, um simulador que possa ser suficientemente acurado, fácil de utilizar 
pelos desenvolvedores de \emph{software} e \emph{hardware} e que possa prover avaliações práticas para novos 
modelos de processamento de sinais e inteligência computacional passa a ter um papel importante 
nesta área da física experimental.


É importante ressaltar que durante 25 anos, o projeto de filtragem \emph{online} proposto pela COPPE/UFRJ 
ficou restrito ao ambiente privado de simulação e reconstrução de eventos da colaboração ATLAS. 
O fato desse ambiente ser rígido e de difícil acesso, e uso restrito à colaboração cientifica que 
desenvolve o ATLAS, impossibilitou diversos estudos (como redes neurais de aprendizado profundo 
ou até mesmo outros algoritmos de extração de \emph{features} do evento). Em maio de 2021, o projeto de 
código aberto \emph{Lorenzetti}\footnote{Nome do simulador de eventos projetado pelo candidato deste memorial.} 
foi registrado na plataforma Zenodo pelo candidato. Em janeiro de 2022, foi lançada a primeira 
versão (DOI 10.5281/zenodo.5884308) estável do ambiente de simulação em colaboração com diversas 
universidades do país (UFBA, UFJF, UFRN, UFRJ e UERJ) e a Universidade \emph{Pierre Marie Curie} (França), 
sendo, hoje, o principal ambiente de estudo de novas tecnologias e provas de conceito do grupo ATLAS/Brazil. Ainda, um artigo científico~\cite{Araujo:2023uwq} sobre este ambiente foi publicado na
\emph{Computer Physics Communication}, tendo o candidato 
como um dos principais autores e colaboradores do projeto.


Além do seu fácil uso e adaptação, proposta diminui drasticamente o processo de avaliação e prototipagem 
das propostas do grupo Brasileiro, possibilitando, inclusive testar outras técnicas antes impossibilitadas 
pelo ambiente oficial da colaboração ATLAS. Após avaliá-las, e eventualmente publicá-las em veículos de 
comunicação científica, as provas de conceito podem ser apresentadas à colaboração ATLAS, onde, eventualmente 
poderão ser implementadas dentro do experimento.


Em março de 2021, após diversas contribuições ao experimento ATLAS, o candidato foi indicado para o cargo de 
coordenação, em conjunto com o pesquisador e professor assistente Christopher Meyer (Universidade de 
Indiana, Estados Unidos) do grupo \emph{Trigger e/gamma}. Sendo, portanto, o primeiro Brasileiro vinculado à uma 
instituição brasileira, em um cargo 
de coordenação, e possivelmente o primeiro a possuir apenas o título de engenheiro, a atuar em um 
grupo pertencente ao sistema de filtragem do ATLAS. Vale ressaltar que este cargo é oferecido apenas para 
professores ou alunos de pós-doutorados associados às principais universidades Europeias e Americanas. 
Ainda, durante o ano de 2021, após o término da bolsa de doutorado oferecida pelo programa de engenharia 
elétrica da UFRJ/CAPES, o candidato passou a integrar o projeto de “diagnóstico auxiliado por computador 
para exclusão de tuberculose ativa em contatos de pacientes com tuberculose pulmonar” que tem como 
coordenadora a Professora Titular Anete Trajman da UFRJ, como bolsista CNPq na modalidade DTI nível A, onde 
realizou contribuições na área de produção de amostras sintéticas de imagens radiológicas do tórax a partir 
de modelos adversariais generativos.



Durante o período de 14 meses, do início do mandato de coordenação do grupo \emph{Trigger e/gamma}, do ATLAS, até 
a data da defesa da tese de doutorado, o candidato exerceu as funções de preparação, organização 
e gerenciamento do grupo de pesquisa internacioal com objetivo de preparar toda a área para o religamento do 
acelerador após 4 anos de parada técnica da fase de \emph{upgrade} do experimento. 
Após defender o trabalho de doutorado~\cite{tese_joao} no 
final de abril de 2022, o candidato iniciou a transferência de obrigações para o pesquisador Brasileiro e 
atual aluno de pós-doutorado, pela universidade \emph{Pierre e Marie Curie}, \emph{Sorbonne}, França, Edmar Egídio 
Purcino de Souza (UFBA/Brasil). Com esse arranjo, o grupo Brasileiro continua com as principais obrigações 
e espaços adquiridos durante os 8 anos de contribuições realizadas pelo candidato, em nome de todo o grupo 
Brasileiro, no \emph{Trigger}. 


Ao final desses quase 8 anos, em números, foram ao todo 457 apresentações (dentre workshops, congressos, reuniões 
de grupo internacional e nacional, apresentações a nível de gerência e organização de mesa de discussão) em 
nome da COPPE/UFRJ e o grupo ATLAS/Brazil, 1 artigo (46 páginas) em fase de revisão referente ao projeto de 
filtragem \emph{online} de elétrons proposto pelo ATLAS, que já tem um \emph{editorial board} do ATLAS e será assinado 
por toda a colaboração (42 países, 3500 pesquisadores), 1 nota técnica (65 páginas) referente a todos os 
resultados obtidos pelo sistema neural nos anos de 2017 e 2018, uma contribuição relevante (uma seção inteira 
sobre as contribuições do projeto do Brasil/COPPE) no artigo \emph{“Performance of Electron and Photon triggers 
in ATLAS during LHC Run 2”}~\cite{paper_egamma_run2} e uma participação no artigo, atualmente em 
preparação, referente aos resultados 
iniciais do \emph{Trigger} de elétrons e fótons durante o primeiro ano da \emph{Run 3} (2022). 
No contexto da contribuição na Rede Nacional de Física de Altas Energias (Renafae), destaca-se a contribuição
do candidato no projeto CAPES COFECUBE, em contexto bilateral Brasil - França, onde o sucesso dos 
desenvolvimentos em sistemas de classificação e o capital político-científico adiquirido 
permitiu extensões para aplicação em Calibração e Regressão de Energia no \emph{Trigger}
Atualmente, o candidato está 
associado, desde dezembro de 2021, ao grupo de pesquisas avançadas e patentes na Dell \emph{Technology Brasil} e foi 
aprovado recentemente (outubro de 2022) no pós-doutorado no PEE/COPPE.



Na Dell, o candidato participa do grupo de pesquisa em inteligência artificial com foco em orquestração 
de \emph{workloads} em infraestrutura computacional heterogênea, computação quântica e segurança. 
Recentemente, o candidato participou do desenvolvimento da prova de conceito do novo sistema de 
orquestração de \emph{workloads} quânticos da empresa intitulado \emph{Quiskit Dell runtime - Intelligent 
Orchestration, which circuit cutting, runtime prediction and resource optimization modules}, onde 
exerceu papel relevante na construção, em conjunto com dois pesquisadores da empresa, do módulo de 
\emph{circuit cutting}. Além desse projeto, o candidato também participou da prova de conceito de orquestração 
de \emph{workloads} utilizando aprendizado por reforço com objetivo de otimizar o tempo de resposta em um 
ambiente de computação heterogêneo, sendo que este último produziu uma patente submetida (\emph{filled}) no 
\emph{United States Patent and Trademark Office} (USPTO). Na área de segurança, o candidato participa de soluções 
utilizando \emph{Knowledge Graph} em infraestrutura computacional com o objetivo de produzir políticas de segurança 
dinâmicas para um ambiente totalmente \emph{Zero-Trust}. Em segurança, o candidato conta atualmente com uma patente 
\emph{filled} no USPTO. Para o pós-doutorado, o candidato aprovou um plano de trabalho onde desenvolverá um modelo 
para geração de imagens sintéticas baseado em modelos generativos até junho de 2023. Concomitantemente, o 
candidato coordenou o desenvolvimento do novo sistema de filtragem \emph{online} de elétrons do ATLAS, comissionado em maio de 2023, 
para \emph{Run 3}.



Além do ramo principal de pesquisa na Física de Partículas e inteligência computacional, o candidato 
também participou como analista \emph{Senior} e coordenador da equipe de ensaios, desde fevereiro de 
2016 até dezembro de 2021, do grupo de avaliação e certificação 
de \emph{software/hardwares} criptográficos alocado no Laboratório de Aplicações Tecnológicas para 
o Setor Produtivo Industrial (LASPI) sob a coordenação do Professor Carlos José Ribas D'Avilla.
Em 2016, o laboratório era autorizado pelo Instituto Nacional de Metrologia, Qualidade e Tecnologia (INMETRO) 
para a avaliação de aparelhos de registro de ponto. 
Em 2017, o candidato foi o responsável técnico pela elaboração do plano de ensaios referente a avaliação 
de módulos criptográficos, com base nos documentos publicados pelo Instituto Nacional de Tecnologia 
da Informação (ITI), INMETRO e o \emph{National Institute of Standards and Technology} (NIST) e pela adequação 
da parte estrutural do laboratório às normas referentes a avaliação de equipamentos de segurança. 
Ainda, naquele mesmo ano, o laboratório foi autorizado pelo ITI, órgão diretamente vinculado a presidência 
da república, após a aprovação do grupo de auditoria do INMETRO, para realizar análise de equipamentos críticos 
de segurança (avaliação de \emph{hardware} e código fonte), sendo atualmente o único laboratório do país 
responsável pela certificação técnica de todos os equipamentos criptográficos (nacionais e internacionais) 
vendidos no Brasil. 

Dentre as atividades realizadas no laboratório pelo candidato destacam-se: A participação ativa como 
analista \emph{Senior} responsável pelo corpo de certificação de módulos críticos de segurança, participação 
em projetos de pesquisa e desenvolvimento realizados em parceria com a Light S/A (através das fundações 
COPPETEC e \emph{Charles Darwin}), responsável pela parte técnica em ambiente de auditoria realizadas anualmente 
pelo INMETRO e coordenador da equipe de avaliação de \emph{software} e \emph{hardware} para módulos criptográfico. 
Em 2020, o candidato também foi responsável pela elaboração do plano de ensaios e homologação 
do código fonte da biblioteca criptográfica, referente as curvas de \emph{Edward} (E521), do Tribunal Superior 
Eleitoral. Em março de 2021, o candidato foi o coordenador do processo de homologação da urna 
eletrônica (modelo UE2020), onde foi o responsável pela avaliação, gerenciamento de equipe e elaboração do relatório referente 
ao desenho do \emph{hardware} e do código fonte do \emph{firmware} do módulo de segurança criptográfico. Em 2021, o 
candidato, também foi o responsável pelo processo de certificação da nova infraestrutura 
de carimbo de tempo, lançada pelo ITI em 2022, e bombas da nova infraestrutura de avaliação de bombas de 
combustível no Brasil.









\subsection{Sobre a Lista de Publicações e Trabalhos}

Esta lista de publicações tem como objetivo enumerar todas as contribuições realizadas pelo o autor ao longo de sua carreira acadêmica desde a iniciação científica  até a conclusão do doutorado pelo Programa de Engenharia Elétrica (PEE) da COPPE/UFRJ e com suporte das agências nacionais de fomento de ensino e pesquisa (CNPq e CAPES). Grande parte dos trabalhos desenvolvidos estão compreendidos na área de inteligência computacional aplicada à física de partículas no experimento ATLAS do \emph{Large Hadron Collider} (LHC).

